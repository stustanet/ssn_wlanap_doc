% Netzwerkanltung für die Studentenstadt Freimann
% Tex initially created by Maximilian Engelhardt <maximilian.engelhardt@stusta.mhn.de>

%\documentclass[a4paper,12pt,draft]{scrartcl}
\documentclass[a4paper,11pt]{scrartcl}

\usepackage[utf8]{inputenc}
\usepackage{ngerman}
\usepackage{eurosym}
%\usepackage{tabularx}
\usepackage[pdftex,final]{graphicx}
\usepackage{wrapfig}
\usepackage[top=1.5cm,bottom=2.5cm,left=1.5cm,right=1.5cm]{geometry}
\pagenumbering{gobble}
\setlength{\parindent}{0in}
%\usepackage[margin=2cm]{geometry}

\usepackage{hyperref}



\title{Kurzanleitung für StuStaNet WLAN-Router\\Quick start Guide for StuStaNet WiFi-router}
\date{}

\begin{document}
	
\maketitle

\begin{figure}[t!]
	\centering
	\vspace{-20pt}
	\includegraphics[width=0.8\textwidth,keepaspectratio]{Bilder/StuStaNet_Logo}
	\vspace{-50pt}
\end{figure}

\vspace{-70pt}

%\section*{Kurzanleitung-\\\Quick start Guide}

\begin{enumerate}
	\item Router anstecken (blaue "`Internet"'-Buchse an (meist) linke Netzwerkdosen an der Wand)\\
	Connect router (blue "'Internet"' socket to the (in most cases) left network wall socket)
	\item Router einschalten (on/off-Knopf hinten)\\
	turn on the router (on/off button at the back)
	\item warte eine Minute, bis der Router hochgefahren ist\\
	wait a minute for the router to boot
	\item den "`WPS"'-Knopf drücken\\
	press the "'WPS"'-Button
	\item warten ca. 2 Minuten\\
	wait for about 2 minutes
	\item Der Router ist nun fertig konfiguriert.\\
	The router is now configured.\vspace{5 mm}
\end{enumerate}

WLAN-Name (SSID) und Passwort stehen auf dem Aufkleber.\\
Wi-Fi name (SSID) and password are on the label.\vspace{5 mm}

Die Einstellungen sind unter \url{http://192.168.1.1} erreichbar.\\
The settings can be accessed using \url{http://192.168.1.1}.\vspace{5 mm}

Um den Router außerhalb der StuSta zu nutzen, muss das Gerät zurückgesetzt werden.\\
In order to use the router outside the StuSta, the device has to be reset.\vspace{5 mm}

Um den Router zurückzusetzen, drücke den "`RESET"'-Knopf für mehr als 10 Sekunden mit einem spitzen Gegenstand.\\
To reset the router, press the "'RESET"'-button for more than 10 seconds with a pointed object\vspace{5 mm}

Weitere Informationen im Wiki: \url{https://wiki.stusta.mhn.de/Wlan}.\\
Further information in the Wiki: \url{https://wiki.stusta.mhn.de/Wlan}.\vspace{5 mm}

Bei weiteren Fragen kann \url{https://stustanet.de/de/support/} weiterhelfen.\\
For further questions \url{https://stustanet.de/en/support/} can be useful.\vspace{5 mm}

\vfill
\begin{center}
	\setlength{\unitlength}{1mm}
	\begin{picture}(70, 35)(0, 0)
	\put(0,0){\line(1,0){4}}
	\put(0,0){\line(0,1){4}}
	\put(70,35){\line(-1,0){4}}
	\put(70,35){\line(0,-1){4}}
	\put(29,17){Aufkleber}
	\put(70,0){\line(-1,0){4}}
	\put(70,0){\line(0,1){4}}
	\put(0,35){\line(1,0){4}}
	\put(0,35){\line(0,-1){4}}
	\end{picture}
\end{center}
\vfill
\flushright\tiny{Version tp-link TL-WR940N-v6.1-2019-04-27}
	
\end{document}