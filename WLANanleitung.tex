% Netzwerkanltung für die Studentenstadt Freimann
% Tex initially created by Maximilian Engelhardt <maximilian.engelhardt@stusta.mhn.de>

%\documentclass[a4paper,12pt,draft]{scrartcl}
\documentclass[a4paper,11pt]{scrartcl}

\usepackage[utf8]{inputenc}
\usepackage{ngerman}
\usepackage{eurosym}
\usepackage{tabularx}
\usepackage[pdftex,final]{graphicx}
\usepackage{wrapfig}
\usepackage[top=1.5cm,bottom=2.5cm,left=1.5cm,right=1.5cm]{geometry}
%\usepackage[margin=2cm]{geometry}

\usepackage{hyperref}



\title{Kurzanleitung für StuStaNet WLAN-Router\\Short manual for StuStaNet wifi-router}
\date{}

\begin{document}

\maketitle

\begin{figure}[t!]
   \centering
   \vspace{-20pt}
   \includegraphics[width=0.8\textwidth,keepaspectratio]{Bilder/StuStaNet_Logo}
   \vspace{-20pt}
\end{figure}


\section*{Kurzanleitung - short manual}

\begin{enumerate}
    \item Router einschalten (schwarzer on/off-Knopf hinten) - turn router on (black switch on the back)
    \item Router anstecken (blaue "`WAN"'-Buchse an Stusta-Netzwerk) - connect router (blue socket to StuSta network socket)
    \item warten, bis der Router gestartet ist (ca. 1 Min) - wait for router to boot (ca. 1 min)
    \item "`WPS/Reset"'-Knopf mehr (!) als 5 sec. gedrückt halten und dann loslassen - push and hold "'WPS/Reset"' button more (!) than 5 seconds
    \item warten ... ca. 2 Min. - wait ... ca. 2 minutes
\end{enumerate}

$\Rightarrow$ Das Gerät ist nun fertig konfiguriert.

$\Rightarrow$ The router is now configured.
\vspace{1em}

WLAN-Name und Passwort stehen auf dem Aufkleber. Der QR-Code kann mit der passenden App gescannt werden. 

Wifi SSID and password are on the label. You can scan the QR-code with a suitable app for easier wifi configuration.

Weitere Informationen im Wiki: \url{https://wiki.stusta.mhn.de/Wlan}.\\[1em]

Bei Fragen und Problemen kann ein Admin konsultiert werden.

If you have problems or questions, ask an Admin.
\vfill
\begin{center}
    \setlength{\unitlength}{1mm}
    \begin{picture}(70, 35)(0, 0)
          \put(0,0){\line(1,0){4}}
          \put(0,0){\line(0,1){4}}
          \put(70,35){\line(-1,0){4}}
          \put(70,35){\line(0,-1){4}}
          \put(29,17){Aufkleber}
          \put(70,0){\line(-1,0){4}}
          \put(70,0){\line(0,1){4}}
          \put(0,35){\line(1,0){4}}
          \put(0,35){\line(0,-1){4}}
    \end{picture}
\end{center}
\vfill

\end{document}

