% Netzwerkanltung für die Studentenstadt Freimann
% Tex initially created by Maximilian Engelhardt <maximilian.engelhardt@stusta.mhn.de>

%\documentclass[a4paper,12pt,draft]{scrartcl}
\documentclass[a4paper,11pt]{scrartcl}

\usepackage[utf8]{inputenc}
\usepackage{ngerman}
\usepackage{eurosym}
%\usepackage{tabularx}
\usepackage[pdftex,final]{graphicx}
\usepackage{wrapfig}
\usepackage[top=1.5cm,bottom=2.5cm,left=1.5cm,right=1.5cm]{geometry}
\pagenumbering{gobble}
\setlength{\parindent}{0in}
%\usepackage[margin=2cm]{geometry}

\usepackage{hyperref}



\title{Kurzanleitung für StuStaNet WLAN-Router\\Quick start Guide for StuStaNet WiFi-router}
\date{}

\begin{document}
	
	\maketitle
	
	\begin{figure}[t!]
		\centering
		\vspace{-20pt}
		\includegraphics[width=0.8\textwidth,keepaspectratio]{Bilder/StuStaNet_Logo}
		\vspace{-50pt}
	\end{figure}
	
	\vspace{-70pt}
	
	%\section*{Kurzanleitung-\\\Quick start Guide}
	
	\begin{enumerate}
		\item Router einschalten (on/off-Knopf hinten)\\
		Turn on the router (on/off button at the back)
		\item Router anstecken (blaue "`Internet"'-Buchse an (meist) linke Netzwerkdosen an der Wand)\\
		connect router (blue "'Internet"' socket to the (in most cases) left network wall socket)
		\item warte eine Minute, bis der Router hochgefahren ist\\
		wait a minute for the router to boot
		\item "`Reset"'-Knopf mit einem spitzen Gegenstand für mehr als 5 Sekunden gedrückt halten\\
		push and hold the "'Reset"' button using a pointed object for more than 5 seconds
		\item warten ca. 2 Minuten, bis die orange LED nicht mehr blinkt\\
		wait about 2 minutes until the orange LED stops flashing
		\item Das Gerät ist nun fertig konfiguriert\\
		The router is now configured\\[1em]
	\end{enumerate}
	
	WLAN-Name (SSID) und Passwort stehen auf dem Aufkleber.
	
	WiFi name (SSID) and password are on the label.
	\\[1em]
	
	Weitere Informationen im Wiki: \url{https://wiki.stusta.mhn.de/Wlan}.
	
	Further information in the Wiki: \url{https://wiki.stusta.mhn.de/Wlan}.\\[1em]
	
	Um den Router außerhalb der StuSta zu nutzen, muss das Gerät in den Einstellungen (\url{http://192.168.1.1}) zurückgesetzt werden.
	
	To use the router outside StuSta, the device has to be reset using the Settings menue (\url{http://192.168.1.1}).\\[1em]
	
	Bei Fragen und Problemen kann ein Admin konsultiert werden.
	
	If you have problems or questions, you can ask an Admin.
	
	\vfill
	\begin{center}
		\setlength{\unitlength}{1mm}
		\begin{picture}(70, 35)(0, 0)
		\put(0,0){\line(1,0){4}}
		\put(0,0){\line(0,1){4}}
		\put(70,35){\line(-1,0){4}}
		\put(70,35){\line(0,-1){4}}
		\put(29,17){Aufkleber}
		\put(70,0){\line(-1,0){4}}
		\put(70,0){\line(0,1){4}}
		\put(0,35){\line(1,0){4}}
		\put(0,35){\line(0,-1){4}}
		\end{picture}
	\end{center}
	%\vfill

	
\end{document}